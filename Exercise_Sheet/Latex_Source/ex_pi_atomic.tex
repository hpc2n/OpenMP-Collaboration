\subsection{Exercise: Calculating Pi} \label{calcPiAtomicExercise}
It can be shown that:
\begin{equation}
\sum_{n=1}^N \frac{1}{n^2} \;\stackrel{N \to \infty}{\longrightarrow}
\;\frac{\pi^2}{6}
\end{equation}

\subsubsection{Serial program}
We provide serial template program in C and Fortran, which sum $1/n^2$ for a large value of $N$.  Experiment with different values of $N$ and make sure your result is correct.

\subsubsection{Parallel program, manual work distribution}\label{PiManual}
Make a copy of the serial template program (or write your own if you prefer) and parallelise it using the OpenMP construct \textit{parallel}.  

In this exercise you should write the code (C or Fortran) to distribute the work onto the threads. That is, starting with the value of $N$, using the number of threads and the thread-id number, each thread should determine the $n$-range it is working on.  For this exercise it is ok, if you assume that $N$ is divisible by the thread count.  Also due to the complexity of this management code, it is ok if your parallel OpenMP version of the  code would not compile in serial.

You will need to introduce additional variables.  Think carefully, which variables need to be declared as \texttt{shared} and which ones need to be declared as \texttt{private}.  We strongly encourage the use of a \verb+default(none)+ clause.  Please do not use any other declarations here.  In this exercise it is easy to create a \textit{data race}.  This has to be avoided while still maintaining high performance (minimise serialisation).

Once the code is finished you should check that 
\begin{enumerate}
\item The parallel code is producing a correct results
\item The results are independent of the thread count
\item Introduce a timer into your code and experiment with different values of $N$.  Rerun you code with 1, 2, 4, 8, 12 and 24 threads.  For small values of $N$ you should notice that the code will run slower for larger thread counts, while for large values of $N$ it will run faster when more threads are used.
\item \label{runtimePi} Do repeated runs of the same run. In particular when using all the threads per  node, do you get the same time?
\end{enumerate}
