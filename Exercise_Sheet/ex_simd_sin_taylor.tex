\subsection{Exercise: Vectorsing code}
Many scientific codes have to evaluate computationally expensive user defined functions.  If these are called inside a loop, the compiler typically required help when vectorising this loop.  In this exercise we will study this in a sample code that calculates the $\sin$ of a vector of lenght 8192.  We have
\begin{equation}
\sin(x) \approx x - \frac{x^3}{3!}  + \frac{x^5}{5!} - \frac{x^7}{7!} + \frac{x^{11}}{11!} + ...
\end{equation}
\subsubsection{Auto vectorisation and compiler feedback}
Engage auto-vectorisation and let the compiler report on vectorisation.  The template code has three loops.  For which loop did the compiler not succeed in vectorising.
\subsubsection{Vectorising the working loop}
In the previous exercise you should have noticed that the compilers (GCC 10.2.0 and Intel 2020.1) can not vectorise the key working loop of the sample code.  Use the OpenMP \verb+simd+ construct to help the compiler.  At this stage you should not use loop constructs.
After vectorisation you should obtain a good performance  boost from the GCC compiler.

\subsubsection{Extra playtime: Combining the loop construct with the {\tt simd} construct}
If you have time left, you can investigate how the SIMD construct and the loop construct work together.   You will need to increase the problem size for this.
